\subsection{管理者側機能}
\subsubsection{ログイン・ログアウト機能}
管理者がGoogle認証を用いて,任意にログイン・ログアウトを行う機能.

ログイン機能では,ログイン画面の「利用規約」をクリックすると利用規約を表示し,
利用規約に同意する場合,利用規約に同意する旨の項目にチェックし,「Googleでログイン」を選択すると,Google認証に移る.
Google認証でアカウントを選択すると,Googleによる認証処理が行われる.認証後,
会員情報を格納しているデータベースにアクセスし,選択したGoogleアカウント情報がデータベースに
登録済みである場合にログインを許可する.
ログイン後,地図表示画面に遷移する.

ログアウト機能では,すでにログインしている管理者が任意のタイミングでログアウト操作を行うことができる.
「ログアウト」を選択すると,セッション情報を無効化し,ログイン画面に遷移する.


\subsubsection{アカウント削除機能}
管理者が本サービスを利用しているユーザーのアカウントを削除することができる機能.
ユーザー管理タブを選択すると,会員情報を格納しているデータベースにアクセスし,遷移した
ユーザー管理画面にユーザー一覧を表示する.ユーザー管理画面に表示されているユーザー一覧の中から
任意のユーザーの欄の右に表示されているゴミ箱マークをクリックすることで,会員情報を格納している
データベースにアクセスし,該当のユーザーの会員情報をデータベースから削除する.
その後,「削除しました!」と表示する.

\subsubsection{事業者アカウントへの変更機能}
管理者が一般会員からの事業者申請を処理する機能.
事業者申請タブを選択すると,申請情報を格納しているデータベースにアクセスし,
申請情報一覧を作成し,事業者申請画面に遷移する.
事業者申請を却下する場合,申請情報を格納しているデータベースにアクセスし,申請情報を削除する.
その後,「却下しました」と表示する.
事業者申請を承認する場合,事業者会員のデータベースにアクセスし,事業者会員インスタンスを追加し,
会員情報に紐づけをする.
その後,「承認しました!」と表示する.

\subsubsection{問い合わせへの対応機能}
管理者が一般会員・事業者会員からの問い合わせへ対応する機能.
問い合わせタブを選択すると,問い合わせ情報を格納しているデータベースにアクセスし,
お問い合わせ画面に遷移する.「未対応」をクリックすると,未対応画面に遷移する.
返信をクリックすると,返信画面が表示され,問い合わせへの返答を行う.
その後,問い合わせ情報が格納されているデータベースにアクセスし,該当の問い合わせを対応済みに
変更し,「対応済みにしました」と表示する.

\subsubsection{運用状況モニタリング機能}
管理者のアカウントの運用状況を可視化した機能.
概要タブを選択すると,総ユーザー数,アクティブユーザー数,総投稿数,事業者アカウント数,
未処理通報数,一週間の新規投稿数,一週間のリアクション数,ジャンルのカテゴリー数を格納した
データベースにアクセスし,概要画面に遷移する.

\subsubsection{通報機能}
管理者が通報された投稿を処理する機能.
通報管理タブを選択すると,通報情報を格納しているデータベースにアクセスし,通報情報を取得し,
通常情報一覧を生成し,通報管理画面に遷移する.
通報を拒否する場合は,通報情報を格納しているデータベースにアクセスし,該当の通報を処理済みに変更し,
「処理しました」と表示する.
承認する場合は,投稿内容を格納しているデータベースにアクセスし,該当の投稿とそれのリアクションを削除し,
「削除しました!」と表示する.



