\subsection{利用者側機能}

\subsubsection{会員機能}
\begin{itemize}   
    \item 新規会員登録
\end{itemize}
一般会員が新規会員登録として,Googleアカウントの情報であるGoogle ID,メールアドレスを
データベースに登録し,会員登録を行う機能.
ログイン画面の「利用規約」をクリックすると,利用規約を表示する.
利用規約に同意する場合,利用規約に同意する旨の項目にチェックし,
「Googleでログイン」を選択すると,Google認証に移る.
Google認証でアカウントを選択すると,Googleによる認証処理が行われる.認証後、
会員情報を格納しているデータベースにアクセスし,選択したGoogleアカウント情報を格納する.
登録後,地図表示画面に遷移する.


\begin{itemize}   
    \item ログイン・ログアウト
\end{itemize}
一般会員がGoogle認証を用いて,任意にログイン・ログアウトを行う機能.

ログイン機能では,ログイン画面の「利用規約」をクリックすると利用規約を表示し,
利用規約に同意する場合,利用規約に同意する旨の項目にチェックし,「Googleでログイン」を選択すると,Google認証に移る.
Google認証でアカウントを選択すると,Googleによる認証処理が行われる.認証後、
会員情報を格納しているデータベースにアクセスし,選択したGoogleアカウント情報がデータベースに
登録済みである場合にログインを許可する.
ログイン後、地図表示画面に遷移する.

ログアウト機能では,すでにログインしている一般会員が任意のタイミングでログアウト操作を行うことができる.
「ログアウト」を選択すると,ログアウト画面に遷移する.ログアウト画面の「キャンセル」を選択した場合,
ログアウト画面へ遷移する前の画面へ遷移する.
「ログアウト」を選択した場合,セッション情報を無効化し,ログイン画面に遷移する.

\begin{itemize}   
    \item 退会
\end{itemize}
一般会員が任意のタイミングで本システムから退会することができる機能.
「マイページ」をクリックし,マイページ画面に遷移する.設定を選択すると「アカウント削除画面へ」
というボタンが表示される.このボタンをクリックすると,退会画面に遷移する.
退会画面では,退会理由(任意)を入力し,削除確認項目にすべてチェックを入れる.
削除確認項目のチェックが不十分な場合,「アカウントを削除する」というボタンにはロックがかかりクリック
することができない.
削除確認項目すべてにチェックを入れ,「キャンセル」を選択した場合,マイページ画面に遷移する.
「アカウントを削除する」を選択した場合,退会確認する旨のメッセージを表示する.
ここで「キャンセル」を選択した場合,退会画面に遷移する.
「OK」を選択した場合,会員情報を格納しているデータベースにアクセスし,会員情報を削除する.
その後,ログイン画面へ遷移する.

\begin{itemize}   
    \item 問い合わせ
\end{itemize}
一般会員が質問や要望をメール形式で運営側に送信することができる機能.
地図表示画面の「お問い合わせ」をクリックすると,お問い合わせ画面に遷移する.
件名・メッセージを入力し,ここで件名・メッセージどちらもまたは片方が未記入の場合,
必須入力が入力されていない旨を表示する.
件名・メッセージ両方入力されている場合は,問い合わせ情報を格納しているデータベースにアクセスし,
問い合わせID,現在の日時,件名,本文(メッセージ),問い合わせユーザーIDを格納する.
その後,「お問い合わせを送信しました。運営からの返信をお待ちください。」と表示する.
運営側からの回答は,一般会員が登録しているGoogleアカウントのメールアドレスへ送信される.

\begin{itemize}   
    \item 事業者申請機能
\end{itemize}
一般会員が事業者会員への昇格を申請するための機能.
地図表示画面の「マイページ」をクリックし,マイページ画面へ遷移する.
マイページ画面に表示される「事業者と登録を申請」を選択すると,事業者登録申請画面へ遷移する.
店舗名の欄に事業者名,電話番号の欄に事業者として使用する電話番号,住所の欄に事業者の住所を
記入した後,「申請」を選択した場合,事業者申請の情報を格納しているデータベースにアクセスし,
申請ID,事業者名,電話番号,住所,申請者ユーザーIDを格納する.
「事業者登録申請を送信しました。運営からの承認をお待ちください。」と表示する.
「申請」を選択せず,「キャンセル」を選択した場合,マイページ画面へ遷移する.


\subsubsection{ピン表示機能}
\begin{itemize}   
    \item 地図上でのピン表示機能
\end{itemize}
他の会員が投稿したピンを地図上に表示する機能.
地図表示画面を表示し,投稿内容を格納しているデータベースにアクセスする.
データベースからマップ上の投稿数を取得し,緯度・経度が同じ地点にある,または「投稿を追加」
から投稿された投稿が50以上の場合,通常のピンの1.3倍の大きさのピンとしてマップ上に描画する.
それ以外の投稿は通常の大きさのピンとしてマップ上に描画される.
また,一般会員による投稿はすべて円形のピンとして描画され,事業者会員による投稿は分に
事業者会員自身が設定したアイコンがピンに描画される.

\begin{itemize}   
    \item ジャンル分け機能
\end{itemize}
投稿時に選択されたジャンルに応じてピンの色を変更する機能.
地図表示画面を表示し,ジャンルの情報を格納しているデータベースにアクセスする.
データベースから表示色を取得し,各ジャンルに割り当てられた固有の色をピンの色に反映する.

\begin{itemize}   
    \item リアクション機能
\end{itemize}
各ピンに対して,他の一般会員がリアクションできる機能.
地図表示画面に描画されたピンを選択しクリックすると,そのピンの投稿内容が表示される.
投稿内容にある「リアクション」を選択すると,投稿内容を格納しているデータベースにアクセスする.
データベースからリアクション数を取得し,「リアクション」を「リアクション済み」に変更,
リアクション数を1足したリアクション数を格納する.
データベースに格納されているリアクション数を反映させる.また,リアクションは匿名で行われる.

\subsubsection{絞り込み機能}
\begin{itemize}   
    \item キーワード検索機能
\end{itemize}
検索窓を使用し,入力したキーワードに応じて情報を検索・絞り込みできる機能.
地図表示画面に表示されている検索窓に検索したいキーワードを入力すると,投稿内容を
格納しているデータベースにアクセスし,データベースの中からキーワードを含む投稿を取得する.
地図表示画面にて、キーワードを含む投稿とピンのみを表示・描画する.投稿は,検索窓の下に
一覧で表示され,ピンは地図上に描画される.

\begin{itemize}   
    \item ジャンル絞り込み機能
\end{itemize}
特定のジャンルに属する投稿・ピンのみを描画する機能.
地図表示画面に表示されているジャンル選択ボタンをクリックし,ジャンルの一覧を表示する.
ジャンルの一覧から任意のジャンルを選択すると,投稿内容とジャンルを格納しているデータベースにアクセスし,
選択されたジャンルの投稿内容を取得し,地図表示画面のジャンル選択ボタンの下に選択したジャンルの投稿のみ
を表示,地図上に選択したジャンルの投稿のピンのみを描画する.

\begin{itemize}   
    \item 期間絞り込み機能
\end{itemize}
指定した日付または期間内に投稿されたピンと投稿のみを表示する機能.
地図表示画面に表示されている期間選択ボタンをクリックし,期間の一覧を表示する.
期間の一覧から任意の期間を選択すると,投稿内容を格納しているデータベースから選択された
期間の投稿内容を取得し,地図表示画面の期間選択ボタンの下に選択した期間の投稿のみが一覧で
表示,地図上に選択した期間の投稿のピンのみを描画する.

\subsubsection{並び替え機能}
\begin{itemize}   
    \item 距離順
\end{itemize}
現在地,または任意の場所から距離の近い順にピンと投稿を並び替えて表示する機能.
地図表示画面に表示されている並び替え選択機能をクリックし,並び替え項目の一覧を表示する.
並び替え項目の一覧から「距離順」を選択すると,投稿内容を格納しているデータベースから投稿内容を
取得し,投稿内容を現在地,または任意の場所から距離の近い順にピンと投稿を並び替える.
地図表示画面の並び替え選択機能の下に並び替えた投稿を表示,地図上には並び替えられた投稿のピン
が描画される.

\begin{itemize}   
    \item リアクション数順
\end{itemize}
リアクションの数が多い順または少ない順にピンと投稿を並び替えて表示する機能.
地図表示画面に表示されている並び替え選択機能をクリックし,並び替え項目の一覧を表示する.
並び替え項目の一覧から「リアクション数順」を選択すると,投稿内容を格納しているデータベースから投稿内容を
取得し,投稿内容をリアクションの数が多い順または少ない順にピンと投稿を並び替える.
地図表示画面の並び替え選択機能の下に並び替えた投稿を表示,地図上には並び替えられた投稿のピン
が描画される.

\begin{itemize}   
    \item 新着順
\end{itemize}
投稿日時が新しい順にピンと投稿を並び替えて表示する機能.
地図表示画面に表示されている並び替え選択機能をクリックし,並び替え項目の一覧を表示する.
並び替え項目の一覧から「新着順」を選択すると,投稿内容を格納しているデータベースから投稿内容を
取得し,投稿内容を投稿日時が新しい順にピンと投稿を並び替える.
地図表示画面の並び替え選択機能の下に並び替えた投稿を表示,地図上には並び替えられた投稿のピン
が描画される.

\subsubsection{ピン投稿機能}
\begin{itemize}   
    \item 場所情報投稿機能
\end{itemize}
一般会員自身がピンとして投稿する機能.
地図表示画面に表示されている「新規投稿」をクリックし,新規投稿画面に遷移する.
新規投稿画面の投稿内容であるタイトル,説明,ジャンル,緯度,経度を入力する.
投稿内容にて写真が任意でアップロードできる.「投稿する」を選択すると,投稿内容を格納している
データベースにアクセスし,投稿内容を格納し,「投稿しました!」という
メッセージを表示する.その後、地図表示画面に遷移し投稿のピンを表示する.
一般会員が投稿した際のピンは共通である.
投稿を取りやめる場合は,新規投稿画面に表示されている「キャンセル」をクリックすると,
地図表示画面に遷移する.

また,新規投稿画面の「投稿内容を入力」の項目について,ピンを表示する位置を入力する必要がある.
位置情報の入力は地図上で選択または,事業者名検索により行う.
位置情報を地図上で選択して入力する場合,新規投稿画面の投稿内容を入力し,「地図上で選択」を選択
すると,地図が表示される.表示された地図の任意の場所をクリックすると,位置情報を取得し,緯度と
経度の項目を入力する.
位置情報を事業者名検索で入力する場合,新規投稿画面の投稿内容を入力し,「事業者名検索」に
事業者名を入力する.事業者会員情報を格納しているデータベースにアクセスし,事業者情報を検索して
入力された事業者名の有無を確認する.検索した事業者名が存在した場合,その事業者の緯度と経度を
入力する.検索した事業者が存在しなかった場合,「その事業者は存在しません」と表示し,新規投稿
画面に遷移する.

\begin{itemize}   
    \item 記述・写真情報投稿機能
\end{itemize}
一般会員自身がつけたピンに対して,説明文等のテキスト,写真情報を投稿できる機能.
地図表示画面に表示されている「新規投稿」をクリックし,新規投稿画面に遷移する.
新規投稿画面の投稿内容で必須入力であるタイトル,説明,緯度,経度を入力する.
必須入力が未記入の場合は、必須入力が未記入である旨を表示する.
任意でジャンルを選択,また写真をアップロードできる.
必須入力がすべて入力されていた場合,投稿内容を格納しているデータベースにアクセスし,
投稿内容を格納する.その後、「投稿しました!」と表示する.

\begin{itemize}   
    \item 時間情報登録機能
\end{itemize}
一般会員がピンの投稿,記述・写真を投稿した際の日時を登録する機能.
新規投稿画面に表示されている「投稿する」を選択したとき,現在日時を取得し,
投稿内容を格納しているデータベースにアクセスし,投稿した日時をデータベースに格納する.

\begin{itemize}   
    \item ジャンル登録機能
\end{itemize}
一般会員が投稿内容のジャンルを登録できる機能.
地図表示画面に表示されている「新規投稿」をクリックし,新規投稿画面に遷移する.
新規投稿画面のジャンル選択ボタンを選択し,ジャンルの項目を展開して表示する.
展開されたジャンルの項目の中から任意のジャンルを選択し,新規投稿画面のその他の必須入力
を入力した上で「投稿する」を選択する.投稿内容を格納しているデータベースにアクセスし,ジャンル
を格納する.また,ジャンルのデフォルトは「その他」となっている.

\begin{itemize}   
    \item 投稿追加機能
\end{itemize}
一般会員がすでに存在しているピンに対して,投稿を追加できる機能.
地図表示画面に表示されている任意のピンを選択すると,そのピンの投稿閲覧画面に遷移する.
「投稿を追加」を選択すると,新規投稿画面に遷移する.
新規投稿画面の必須入力であるタイトル,説明,緯度,経度を入力し,また必須入力が未記入の場合は
必須入力が未記入である旨を表示する.
必須入力を入力し,「投稿する」を選択すると,投稿内容を格納しているデータベースにアクセスし,
投稿内容を格納する.その際に、ピンに対する投稿数を1増やす.ピンに対する投稿数が50以上ならば
ピンの大きさを通常の1.3倍の大きさにして地図表示画面の地図上に表示するようにする.
ピンに対する投稿数が50未満の場合は,ピンの大きさは通常の大きさで地図表示画面の地図上に表示
する.

\subsubsection{マイページ機能}
\begin{itemize}   
    \item リアクション履歴閲覧機能
\end{itemize}
一般会員自身がこれまでにリアクションを行った投稿の履歴を閲覧できる機能.
「マイページ」を選択すると,マイページ画面に遷移する.
マイページ画面のタブ内にある「リアクション履歴」を選択すると,リアクションの情報を格納している
データベースにアクセスし,リアクションを取得することで,一般会員自身のリアクション履歴を表示する.

\begin{itemize}   
    \item 投稿履歴閲覧機能
\end{itemize}
一般会員自身がこれまでに投稿したピン投稿の履歴を一覧で確認できる機能.
「マイページ」を選択すると,マイページ画面に遷移する.
マイページ画面のタブ内にある「投稿履歴」を選択すると,投稿内容を格納しているデータベースにアクセスし,
投稿内容を取得することで,一般会員自身の投稿履歴を表示する.

\begin{itemize}   
    \item 投稿内容削除機能
\end{itemize}
一般会員自身がこれまでに投稿したものを削除できる機能.
「マイページ」を選択すると,マイページ画面に遷移する.
マイページ画面のタブ内にある「投稿履歴」を選択すると,投稿内容を格納しているデータベースにアクセスし,
投稿内容を取得することで,一般会員自身の投稿履歴を表示する.
表示された投稿履歴の任意の投稿の横にあるゴミ箱マークを選択すると,投稿削除を確認する旨が表示される.
「キャンセル」を選択した場合,マイページ画面に遷移する.「OK」を選択した場合,
投稿内容を格納しているデータベースにアクセスし,任意の投稿内容を削除する.
その後,「投稿を削除しました」と表示する.

\subsubsection{通報機能}
一般会員が不適切な投稿を運営へ通知するための機能.
投稿閲覧画面で「通報」を選択すると,通報理由の入力フォームが表示され,通報理由を入力する.
「通報する」を選択したとき,通報理由が入力されていない場合,通報理由が入力されていない旨を表示し、
通報理由の入力フォームが表示される.通報理由が入力されていた場合,投稿内容の通報の情報を格納する
データベースにアクセスし,通報ID,通報したユーザーのユーザーID,対象投稿ID, 通報理由,通報日時を
格納する.その後,「通報を受け付けました。運営が確認いたします。」と表示する.

\subsubsection{ブロック機能}
\begin{itemize}   
    \item ブロック機能
\end{itemize}
一般会員が特定の利用者に対して表示制限を行うための機能.
投稿閲覧画面の「ブロック」を選択すると,「このユーザーをブロックしますか?ブロックする
と相手の投稿が表示されなくなります。」と表示される.「キャンセル」を選択した場合,
投稿閲覧画面に遷移する.「OK」を選択した場合,会員情報を格納しているデータベースから該当の会員情報
を取得し,またブロックの情報を格納しているデータベースに取得した会員情報をブロック情報として格納する.
その後,「ユーザーをブロックしました」と表示し,地図表示画面に遷移する.

\begin{itemize}   
    \item ブロック解除機能
\end{itemize}
一般会員が特定の利用者に対して表示制限を行う機能.
マイページ画面の設定のタブをクリックすると,ブロックリストが表示される.
任意のユーザーの「ブロック解除」を選択すると,ブロックの情報を格納したデータベースにアクセスし,
任意のユーザーIDを削除する.その後,ブロックリストからユーザーIDを消去する.


