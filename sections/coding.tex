以下に本プロジェクトで採用するコーディング規約を示す.



\subsection{HTML}
以下に本システムに用いられたHTMLファイルに適用されるコーディング規約を示す.

\subsubsection{命名規約}
\begin{itemize}
    \item ファイル名
    \begin{itemize}
        \item アッパーキャメルケース(例:CamelCase)表記を使用する
        \item シンプルで内容を示す英単語で構成する
    \end{itemize}    
\end{itemize}

\subsubsection{ドキュメント構造}
全てのHTMLファイルは以下の構造に従う.

\begin{verbatim}
<!DOCTYPE html>
<html lang="ja">
    <head>
        <meta charset="utf-8" />
        <meta name="viewport" content="width=device-width, initial-scale=1.0" />
        <title>ページのタイトル</title>
        <link rel="stylesheet" href="styles.css" />
    </head>
    <body>
        <!-- ページのコンテンツ -->

        <script src="main.js"></script>
    </body>
</html>
\end{verbatim}

    \begin{itemize}
        \item DOCTYPE 宣言を行う.
        \item コメントアウトは「\verb|<!-- コメントアウト -->」で書く
        
    \end{itemize}



\subsection{CSS}
以下に本システムに用いられたCSSファイルに適用されるコーディング規約を示す.

\subsubsection{命名規約}
\begin{itemize}
    \item クラス名
    \begin{itemize}
        \item 役割が分かるように命名する
        \item ケバブケース(小文字,ハイフン)表記を使用する
    \end{itemize}  
\end{itemize}  

\subsubsection{インデント}
\begin{itemize}
        \item タブの使用は禁止する
        \item スペース2つで対応する
\end{itemize} 

\subsubsection{コメント規約}
\begin{itemize}
  \item セクションコメント
    \begin{itemize}
        \item ブロックコメントでセクション名を明記する
        \item 上下に装飾線を入れる
   \end{itemize}  
  \item 行コメント
   \begin{itemize}
    \item ブロックコメントで補足などを書く
   \end{itemize}
\end{itemize}
  
  





\subsection{Go}
以下に本システムに用いられたGoファイルに適用されるコーディング規約を示す.

\subsubsection{命名規約}
\begin{itemize}
    \item パッケージ名
    \begin{itemize}
        \item 小文字の単語で簡潔に表記する
    \end{itemize}  
    \item 関数名・構造体
    \begin{itemize}
       \item 外部に公開する場合,先頭を大文字にする
       \item アッパーキャメルケースとキャメルケースを使い分ける
       \item アンダースコアは使用しない
    \end{itemize}  
    \item インターフェース名
    \begin{itemize}
        \item 単一メソッドインターフェースは-er接尾辞を使用する
    \end{itemize}  
      
\end{itemize}

\subsubsection{インデント}
\begin{itemize}
        \item タブを使用する
        \item 読みにくくなる長文は改行する
        \item 演算子の前後には空白を入れる
\end{itemize}   

\subsubsection{エラーハンドリング}
以下にエラーを返す関数を示す.
\begin{verbatim}
func readFile(name string) (string, error) {
    data, err := os.ReadFile(name)
    if err != nil {
        return "", err
    }
    return string(data), nil
}
\end{verbatim}
戻り値にerrorを返す.

\subsubsection{コメント規約}
\begin{itemize}
        \item 公開要素には必ずコメントを書く
        \item 関数名から書き始める
\end{itemize}   

\subsubsection{jsonタグ}
\begin{itemize}
  \item jsonタグは必ず明記する
  \item スネークケース表記を使用する
\end{itemize}  






\subsection{Typescript}
以下に本システムに用いられたTypescriptファイルに適用されるコーディング規約を示す.

\subsubsection{命名規約}
\begin{itemize}
    \item ファイル名
    \begin{itemize}
        \item アッパーキャメルケース(例:CamelCase)表記を使用する
        \item シンプルで内容を示す英単語で構成する
    \end{itemize}   
    \item 変数名
    \begin{itemize}
        \item アッパースネークケース(例:UPPER\_SNAKE\_CASE)表記を使用する
    \end{itemize}   
\end{itemize}

\subsubsection{インデント}
\begin{itemize}
        \item タブの使用は禁止する
        \item スペース2つで対応する
        \item 演算子の前後には空白を入れる
        \item カンマ,if,forの後ろも同様に空白を入れる
    \end{itemize}   

\subsubsection{コメント規約}
\begin{itemize}
        \item (例://情報を初期化)のようになにをしているかを明記する
    \end{itemize}   










